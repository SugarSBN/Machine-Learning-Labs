\documentclass{article}
\usepackage{fontspec}
\setmainfont{Times New Roman}
\usepackage{geometry}
\usepackage{CTEX}
\geometry{papersize={21cm,29.7cm}}
\geometry{left=3.18cm,right=3.18cm,top=2.54cm,bottom=2.54cm}
\usepackage{fancyhdr}
\usepackage{amsmath}
\pagestyle{fancy}
\lhead{学号:202000460020}
\rhead{姓名:苏博南}
\cfoot{\thepage}
\renewcommand{\headrulewidth}{0.4pt}
\renewcommand{\headwidth}{\textwidth}
\usepackage{tikz}
\usetikzlibrary{automata, positioning, arrows}
\usepackage{listings}
\usepackage{float}

\newtheorem{question}{题目}  
\lstset{
	basicstyle=\small\ttfamily,	% 基本样式
		keywordstyle=\color{blue}, % 关键词样式
		commentstyle=\color{gray!50!black!50},   	% 注释样式
		stringstyle=\rmfamily\slshape\color{red}, 	% 字符串样式
	backgroundcolor=\color{gray!0},     % 代码块背景颜色
	frame=leftline,						% 代码框形状
	framerule=12pt,%
		rulecolor=\color{gray!0},      % 代码框颜色
	numbers=left,				% 左侧显示行号往左靠, 还可以为right ,或none,即不加行号
		numberstyle=\footnotesize\itshape,	% 行号的样式
		firstnumber=1,
		stepnumber=1,                  	% 若设置为2,则显示行号为1,3,5
		numbersep=7pt,               	% 行号与代码之间的间距
	aboveskip=.25em, 			% 代码块边框
	showspaces=false,               	% 显示添加特定下划线的空格
	showstringspaces=false,         	% 不显示代码字符串中间的空格标记
	keepspaces=true, 					
	showtabs=false,                 	% 在字符串中显示制表符
	tabsize=2,                     		% 默认缩进2个字符
	captionpos=b,                   	% 将标题位置设置为底部
	flexiblecolumns=true, 			%
	breaklines=true,                	% 设置自动断行
	breakatwhitespace=false,        	% 设置自动中断是否只发生在空格处
	breakautoindent=true,			%
	breakindent=1em, 			%
	title=\lstname,				%
	escapeinside=``,  			% 在``里显示中文
	xleftmargin=1em,  xrightmargin=1em,     % 设定listing左右的空白
	aboveskip=1ex, belowskip=1ex,
	framextopmargin=1pt, framexbottommargin=1pt,
        abovecaptionskip=-2pt,belowcaptionskip=3pt,
	% 设定中文冲突,断行,列模式,数学环境输入,listing数字的样式
	extendedchars=false, columns=flexible, mathescape=true,
	texcl=true,
	fontadjust
}%

\begin{document}

\begin{center}
    \huge{机器学习课程实验六}\\
    \large{\today \quad 苏博南\quad 202000460020}
\end{center}

首先先贴老师原话:
\begin{figure}[H]
    \centering
    \includegraphics[width=\linewidth]{15.png}
\end{figure}

然后一看代码,PCA部分不是import自sklearn.decomposition就是import自playML。
好嘛那我就贴结果和感受了。
\section{结果}
\begin{figure}[H]
    \centering
    \includegraphics[width=\linewidth]{1.png}
\end{figure}
\begin{figure}[H]
    \centering
    \includegraphics[width=\linewidth]{2.png}
\end{figure}
\begin{figure}[H]
    \centering
    \includegraphics[width=\linewidth]{3.png}
\end{figure}
\begin{figure}[H]
    \centering
    \includegraphics[width=\linewidth]{4.png}
\end{figure}
\begin{figure}[H]
    \centering
    \includegraphics[width=\linewidth]{5.png}
\end{figure}
\begin{figure}[H]
    \centering
    \includegraphics[width=\linewidth]{6.png}
\end{figure}
\begin{figure}[H]
    \centering
    \includegraphics[width=\linewidth]{7.png}
\end{figure}
\begin{figure}[H]
    \centering
    \includegraphics[width=\linewidth]{8.png}
\end{figure}
\begin{figure}[H]
    \centering
    \includegraphics[width=\linewidth]{9.png}
\end{figure}
\begin{figure}[H]
    \centering
    \includegraphics[width=\linewidth]{10.png}
\end{figure}
\begin{figure}[H]
    \centering
    \includegraphics[width=\linewidth]{11.png}
\end{figure}
\begin{figure}[H]
    \centering
    \includegraphics[width=\linewidth]{12.png}
\end{figure}
\begin{figure}[H]
    \centering
    \includegraphics[width=\linewidth]{13.png}
\end{figure}
\begin{figure}[H]
    \centering
    \includegraphics[width=\linewidth]{14.png}
\end{figure}

\section{感受}

没有啥感受。不知道代码在干嘛,反之运行了一遍,也不用写PCA,KNeighboursClassifier也不知道在干嘛,认为实验设置不合理,
对非机器学习专业同学不友好,体验挺差,不如之前写个pdf用matlab求一下协方差和本征矢更像回事,或者用spss因子分析看一下相关系数,
总之感觉python确实意义不明了。
\end{document}